\documentclass{article}
\usepackage[utf8]{inputenc}
\usepackage{changepage}% http://ctan.org/pkg/changepage
\usepackage{float}
\usepackage{fancyhdr}
\usepackage{lastpage}
\usepackage{graphicx}
\usepackage{ragged2e}
\usepackage{scrextend}
\usepackage{lastpage}
\pagestyle{fancy}
\renewcommand{\headrulewidth}{0pt}
\rhead{\includegraphics[height= 1cm]{ento1.png}\vspace{1em}}
\chead{\includegraphics[width = 5cm]{agritech1.png}}
\lhead{\includegraphics[width=3cm]{acet1.png}}
 
\fancyfoot{}
\setlength\headheight{40.8pt}
\RequirePackage[colorlinks=true, allcolors=blue]{hyperref}
\usepackage{titlesec}
\titleformat{\section}{\normalfont\fontsize{10}{15}\bfseries}{\thesection}{0em}{}
\titlespacing{\section}{}{1.24em}{0.24em}
 
 
\cfoot{Page \thepage \hspace{1pt} of \pageref{LastPage}}

\begin{document}

\begin{addmargin}[2.8in]{}
Willett, Filgueiras, \newline
Benda, Zhang, and Kenworthy \newline
----------------------------- \newline
Denis S. Willett\newline
Department of Entomology \newline
Cornell AgriTech  \newline
Cornell University \newline
15 Castle Creek Drive \newline
Geneva, NY 14456 \newline
deniswillett@cornell.edu \newline
\end{addmargin}
\setlength{\parindent}{0cm}

April 29, 2019

\vspace{1.24em}

Nature Plants

\vspace{1.24em}

Dear Editors,

\vspace{0.48em}
\setlength{\parindent}{1.24cm}

Enclosed please find an electronic copy of the manuscript entitled \textit{“Sting nematodes modify metabolomic profiles of host plants”} for consideration in Nature Plants.  In this manuscript, we present novel data showing that nematode ectoparasites of plants can modify host metabolomes.  While previous work on root-knot and cyst nematode has demonstrated that nematode endoparasites of plants alter metabolite expression, this is the first work documenting a migratory ectoparasite (with a very different lifestyle and co-evolutionary history) extensively modifying host metabolomes.  Specifically, sting nematodes alter global metabolomic profiles in susceptible, moderately tolerant, and tolerant African bermudagrass cultivars through suppression of amino acids (in contrast to endoparasites which upregulate amino acid production) and stimulation of plant defense pathways.  We show that pipecolic acid, a systemic acquired resistance regulator, likely plays a key role in mediating tolerance to sting nematode infection.  


The results of this work are timely and of likely interest to a broad audience. The role of pipecolic acid in regulating systemic acquired resistance is only just being discovered.  Here, we present evidence showing that it likely plays a large role as a mechanism of nematode tolerance.  Additionally, we present data showing that despite suppression of amino acid production, key members of the salicylic acid pathway are possible regulators of nematode ectoparasite infection ability.  Not only does this work point to metabolic mecahnisms of tolerance to nematode infection of plants, this work also informs theory underlying our understanding of plant parasitic nematode-plant interactions.  Migratory ectoparasitic nematodes like the sting nematode have often been considered somewhat opportunistic feeders on host plants with little co-evolution necessary; if a food source was not suitable, migratory ectoparasitic nematodes could go elsewhere.  Endoparasitic nematodes, on the other hand, have long been recognized to have an extensive co-evolutionary history with the host plants and extensively modify the host plant metabolome through creation of feeding cells.  Our work indicates that despite their ability to move elsewhere, migratory ectoparasites can extensively modify the metabolome of their host plants and hints at a longer co-evolutionary relationship than that assumed previously.  Researchers interested in plant-pathogen relationships, mechanisms of tolerance, and metabolites involved in systemic acquired resistance will likely find this manuscript of interest.  


This manuscript may likely be of interest to Guillaume Tena as a possible suggested associate editor. The authors have not had prior discussions with the Nature Plants Editorial Board Members about this work.  We suggest the following experts in the field as possible peer reviewers:


\setlength{\parindent}{0cm}
\begin{addmargin}[1in]{}

Dr. Larry W Duncan \newline
Professor | Department of Entomology and Nematology \newline
University of Florida \newline
lwduncan@ufl.edu \newline

Dr. Baldwyn Torto \newline
Principal Scientist | International Center of Insect Physiology and Ecology \newline
Nairobi, Kenya \newline
btorto@icipe.org \newline

Dr. Tesfamariam Mengistu \newline
National Program Leader | Division of Plant Systems-Protection \newline
NIFA USDA \newline
tesfamariam.mengistu@usda.gov \newline


Thank you in advance for your consideration.  

\vspace{2em}

Much appreciated, \newline

\vspace{1em}

Denis S Willett
\vspace{0.48em}

Camila Cramer Filgueiras
\vspace{0.48em}

Nicole D Benda
\vspace{0.48em}

Jing Zhang
\vspace{0.48em} 

Kevin E Kenworthy


\end{document}

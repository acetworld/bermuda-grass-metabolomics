\documentclass{article}
\usepackage[utf8]{inputenc}
\usepackage{changepage}% http://ctan.org/pkg/changepage
\usepackage{float}
\usepackage{fancyhdr}
\usepackage{lastpage}
\usepackage{graphicx}
\usepackage{ragged2e}
\usepackage{scrextend}
\usepackage{lastpage}
\pagestyle{fancy}
\renewcommand{\headrulewidth}{0pt}
\rhead{\includegraphics[height= 1cm]{ento1.png}\vspace{1em}}
\chead{\includegraphics[width = 5cm]{agritech1.png}}
\lhead{\includegraphics[width=3cm]{acet1.png}}
 
\fancyfoot{}
\setlength\headheight{40.8pt}
\RequirePackage[colorlinks=true, allcolors=blue]{hyperref}
\usepackage{titlesec}
\titleformat{\section}{\normalfont\fontsize{10}{15}\bfseries}{\thesection}{0em}{}
\titlespacing{\section}{}{1.24em}{0.24em}
 
 
\cfoot{Page \thepage \hspace{1pt} of \pageref{LastPage}}

\begin{document}

\begin{addmargin}[2.8in]{}
Willett, Filgueiras, \newline
Benda, Zhang, and Kenworthy \newline
----------------------------- \newline
Denis S. Willett\newline
Department of Entomology \newline
Cornell AgriTech  \newline
Cornell University \newline
15 Castle Creek Drive \newline
Geneva, NY 14456 \newline
deniswillett@cornell.edu \newline
\end{addmargin}
\setlength{\parindent}{0cm}

October 24, 2019

\vspace{1.24em}

Scientific Reports

\vspace{1.24em}

Dear Dr. Geary,

\vspace{0.48em}
\setlength{\parindent}{1.24cm}

Thank you for your consideration of \textit{“Sting nematodes modify metabolomic profiles of host plants”} for publication in Scientific Reports.  We have revised the manuscript in accordance with reviewer recommendations.  A complete and detailed account of the revisions is included below.  The reviewer comments were very helpful, much appreciated, and have resulted in a better manuscript.  In addition to submitting a revised version of the manuscript incorporating the latest changes, we are also submitting production quality images.  
  

\vspace{2em}

Much appreciated, \newline

\vspace{1em}

Denis S Willett
\vspace{0.48em}

Camila Cramer Filgueiras
\vspace{0.48em}

Nicole D Benda
\vspace{0.48em}

Jing Zhang
\vspace{0.48em} 

Kevin E Kenworthy

\vspace{1.4em}

\hline

\setlength\parindent{0pt}

\subsection*{Reviewer 1 (Remarks to the Author)}

You spend a lot of time talking about relative tolerance of the bermudagrass lines to sting nematode. However, I was never very clear on how you are defining tolerance or how this was determined.

\begin{quote}
    \textit{This is an excellent question.  We have taken the opportunity to elaborate on this in the introduction.  In short, tolerance in bermudagrass is determined if there was no reduction in root length or had greater root length than ‘Tifway’ despite \textit{B. longicaudatus} infection.  Tolerance of the varieties had been established in previous work by Pang et al. 2011, citations 23, and 24 in the manuscript.  } 
\end{quote}

On page 2 under Bermudagrass response to sting nematode feeding...I think you should use the term nematode population densities instead of "nematode densities" or "differences in nematode population."

\begin{quote}
    \textit{Thanks for pointing this out.  We have adjusted the language in that section in accordance with your suggestions.} 
\end{quote}

Very well written and very interesting paper

\begin{quote}
    \textit{Thank you for your interest and constructive feedback.}
\end{quote}


\subsection*{Reviewer #2 (Technical Comments to the Author)}

This is a well researched study and the manuscript well structured and written. The methods used are scientifically sound, results appropriately interpreted and discussed. Except for a few minor comments below, I recommend publication of this manuscript.

\begin{quote}
    \textit{Thank you for your feedback.  We appreciate the interest and comments below.}
\end{quote}

Line 58. The authors stated that the canonical correspondence analysis explained 42\% of the observed inertia in the total metabolome. I suggest that the authors explain this because in Figure 2 this analysis explained 71.9\% of the total metabolome variation (CCA1: 45.3\% + CCA2: 26.6\%).

\begin{quote}
  \textit{Thanks for your inquiry.  Canonical correspondence analysis did explain 42\% of the observed inertia.  Of that 42\% explained inertia, axis 1 explained 45.3\% and axis 2 26.6\%.  Summing the explained variation of the other 3 axis with the first two would result in 100\% of the variation captured in canonical ordination or 42\% of observed inertia (constrained $\chi^2$) in the total metabolome. This explanation has been elaborated upon in the text.   } 
\end{quote}

Lines 146 and 150. Change ‘conetainers’ to ‘containers and container’

\begin{quote}
    \textit{Thanks for suggesting this. Conetainers should be Cone-tainers to refer to the specific type of container used to house the plants.  We have clarified this in the Methods section adding both information about the company from which they can be procured and changing the spelling in lines 148 and 150 for clarification.  }
\end{quote}

I suggest that instead of putting in the appendix, the package names used for their statistical analysis, the authors should include these names in the text to educate readers, especially for those not familiar with the ‘R software’

\begin{quote}
    \textit{Thanks for your suggestion.  This is a good idea; we have incorporated names of the software packages in the main body of the text.}
\end{quote}

\end{document}

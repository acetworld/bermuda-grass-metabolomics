%%%%%%%%%%%%%%%%%%%%%%%%%%%%%%%%%%%%%%%%%%%%%%%%%%%%%%%%%%%%
%%% ELIFE ARTICLE TEMPLATE
%%%%%%%%%%%%%%%%%%%%%%%%%%%%%%%%%%%%%%%%%%%%%%%%%%%%%%%%%%%%
%%% PREAMBLE 
\documentclass[9pt,lineno]{elife}
% Use the onehalfspacing option for 1.5 line spacing
% Use the doublespacing option for 2.0 line spacing
% Please note that these options may affect formatting.
% Additionally, the use of the \newcommand function should be limited.


\usepackage{lipsum} % Required to insert dummy text
\usepackage[version=4]{mhchem}
\usepackage{siunitx}
\DeclareSIUnit\Molar{M}

%%%%%%%%%%%%%%%%%%%%%%%%%%%%%%%%%%%%%%%%%%%%%%%%%%%%%%%%%%%%
%%% ARTICLE SETUP
%%%%%%%%%%%%%%%%%%%%%%%%%%%%%%%%%%%%%%%%%%%%%%%%%%%%%%%%%%%%
\title{This is the title}

\author[1*]{Firstname Middlename Surname}
\author[1,2\authfn{1}\authfn{3}]{Firstname Middlename Familyname}
\author[2\authfn{1}\authfn{4}]{Firstname Initials Surname}
\author[2*]{Firstname Surname}
\affil[1]{Institution 1}
\affil[2]{Institution 2}

\corr{email1@example.com}{FMS}
\corr{email2@example.com}{FS}

\contrib[\authfn{1}]{These authors contributed equally to this work}
\contrib[\authfn{2}]{These authors also contributed equally to this work}

\presentadd[\authfn{3}]{Department, Institute, Country}
\presentadd[\authfn{4}]{Department, Institute, Country}
% \presentadd[\authfn{5}]{eLife Sciences editorial Office, eLife Sciences, Cambridge, United Kingdom}

%%%%%%%%%%%%%%%%%%%%%%%%%%%%%%%%%%%%%%%%%%%%%%%%%%%%%%%%%%%%
%%% ARTICLE START
%%%%%%%%%%%%%%%%%%%%%%%%%%%%%%%%%%%%%%%%%%%%%%%%%%%%%%%%%%%%

\begin{document}

\maketitle

\begin{abstract}
Please provide an abstract of no more than 150 words. Your abstract should explain the main contributions of your article, and should not contain any material that is not included in the main text.
\end{abstract}


\section{Introduction (Level 1 heading)}

Thanks for using Overleaf to write your article. Your introduction goes here! Some examples of commonly used commands and features are listed below, to help you get started.

Here's a second paragraph to test paragraph indents. \lipsum[1]

\section{Results}


Fig1

Fig2

Fig3

Fig4






\lipsum[2-3]

\begin{table}[bt]
\caption{\label{tab:example}Automobile Land Speed Records (GR 5-10).}
% Use "S" column identifier to align on decimal point 
\begin{tabular}{S l l l r}
\toprule
{Speed (mph)} & Driver          & Car                        & Engine    & Date     \\
\midrule
407.447     & Craig Breedlove & Spirit of America          & GE J47    & 8/5/63   \\
413.199     & Tom Green       & Wingfoot Express           & WE J46    & 10/2/64  \\
434.22      & Art Arfons      & Green Monster              & GE J79    & 10/5/64  \\
468.719     & Craig Breedlove & Spirit of America          & GE J79    & 10/13/64 \\
526.277     & Craig Breedlove & Spirit of America          & GE J79    & 10/15/65 \\
536.712     & Art Arfons      & Green Monster              & GE J79    & 10/27/65 \\
555.127     & Craig Breedlove & Spirit of America, Sonic 1 & GE J79    & 11/2/65  \\
576.553     & Art Arfons      & Green Monster              & GE J79    & 11/7/65  \\
600.601     & Craig Breedlove & Spirit of America, Sonic 1 & GE J79    & 11/15/65 \\
622.407     & Gary Gabelich   & Blue Flame                 & Rocket    & 10/23/70 \\
633.468     & Richard Noble   & Thrust 2                   & RR RG 146 & 10/4/83  \\
763.035     & Andy Green      & Thrust SSC                 & RR Spey   & 10/15/97\\
\bottomrule
\end{tabular}

\medskip 
Source: \url{https://www.sedl.org/afterschool/toolkits/science/pdf/ast_sci_data_tables_sample.pdf}

\tabledata{This is a description of a data source.}

\end{table}

\subsection{Level 2 Heading}

\lipsum[3]

\subsubsection{Level 3 Heading}

\lipsum[5]

\paragraph{Level 4 Heading}
\lipsum[7]

\section{Discussion}

\lipsum[9]

\section{Methods and Materials}


\subsection{Organisms}

Three African Bermudagrass (\textit{Cynodon transvaalensis} Burtt-Davy) lines were used to evaluate metabolomic responses to pathogen infection.  One susceptible line (AB03), one tolerant line (AB39), and one line with moderate tolerance (AB33) were stolon propagated and allowed to establish for 30 days in 4cm diameter, 30cm deep UV-stabilized Ray Leach Cone-tainers filled with USGA grad sand followed with a soil plug (to prevent sand leakage). Plants were grown in a climate controlled greenhouse at 25C and 50\% RH under a 11:13 light:dark cycle.  Plants were watered twice daily for 5 minutes and received 24-8-16 NPK liquid fertilizer on a weekly basis.  Sting nematode (\textit{Belonolaimus longicaudatus} Rau) originally collected from Florida turf and reared on \textit{C. transvaalensis} were used in bioassays. 

\subsection{Bioassays}

To evaluate the effects of nematode infection on bermudagrass metabolite response, treated plants in conetainers were inoculated with 50 sting nematodes of mixed age and gender per conetainer.  Control (uninoculated) plants did not receive any nematode treatment. Seven replications of each line (AB03, AB33, AB39) and treatment (inoculated, uninoculated) combination were conducted.  Ninety days following nematode inoculation (time enough for nematode growth and reproduction), plants were removed from the conetainer, the soil plug removed, and the sand gently rinsed to extract the nematodes for further processing through centrifugal flotation and counting on an inverted microscope.  Rinsed roots were placed in 50ml falcon tubes then immersed in liquid nitrogen and stored at -80 C until lyophilization.  Lyophilzed roots were weighed and samples selected for metabolomics analysis. 

\subsection{Metabolomics}

Following weighing the entire root biomass, 0.5 gram root samples were transferred to 2ml centrifuge tubes and ground in a Geno/Grinder 2010 tissue homogenizer with ball bearings.  After grinding, metabolites were extracted through addition of 1.5ml of 1:1 Methanol:Ammonium Acetate, addition of 20ul internal standard mix, vortexing, and centrifugation at 17,000G for 10 minutes.  Following centrifugation, 800ul of supernatant was transfered to an LC vial and 1ul introduced to a Thermo Scientific Dionex Ultimate 3000 UHPLC using reverse phase chromatography with a ACE Excel 2 C18-PFP (100 x 2.1mm, 2um) at 25C and a flow rate of 350ul/min.  ....Solvent Gradient Here.... 

Following separation by liquid chromatography, samples were introduced to a Thermo Q-Exactive Orbitrap mass spetrometer.  All samples were analyzed in positive and negative heated electrospray ionization with a mass resolution of 35,000 at m/z 200 using polarity switching....More details.  


\subsection{Analysis}

\subsubsection{Bioassays}

The effects of bermudagrass line, treatment, and their interaction on observed nematode counts were model using linear models and analysis of variance.  Residual diagnostics were consulted to ensure conformity of assumptions of normality and homoscedasticity while model significance, likelihood ratios, information criteria, coefficient of determination, and residual examination were used to select the best fit models.  Post-hoc comparisons were evaluated using Tukey's method for controlling the family-wise error rate. 

Similarly, the effects of bermudagrass line, treatment, nematode count, and their interaction on observed root weights were modeled using linear models and analysis of variance. Residual diagnostics were consulted to ensure conformity of assumptions of normality and homoscedasticity while model significance, likelihood ratios, information criteria, coefficient of determination, and residual examination were used to select the best fit models.  Outliers were identified and removed through visual examination of residual diagnostics (including QQ-Plots, Cook's Distance, and Leverage) and mean-shift outlier tests.  Post-hoc comparisons were evaluated using Tukey's method for controlling the family-wise error rate.

Root biomass loss estimation was accomplished through non-parametric bootstrapping (with 1000 replications) to estimate the difference in root biomass between bermudagrass plants inoculated with sting nematode and uninoculated plants.  Differences between median root loss were evaluated with one-sided permutation tests and adjusted using Bonferroni's method for controlling the family-wise error rate.


\subsubsection{Metabolomics}

Raw mass spectrometry data were exported and uploaded to Metabolomics Workbench (Study ID ST000353).  In preparation for analysis, known compounds with more than one retention time were collapsed into a single known compound.  Additionally, contaminants and internal standards were removed from future analysis (Appendix A for list of contaminants and standards removed).  Following cleaning, missing data (less than 8.3\% per sample) were imputed using a k-nearest neighbors approach (k = 5).  Following imputation, data were normalized using variance stabilizing normalization (li et al) to adjust for between-run variations.  

To determine whether there were differences between line and nematode treatment, canonical correspondence analysis was applied to the global metabolome (both known and unidentified compounds).  Results from the canonical correspondence analysis were further evaluated with permutational analysis of variance with 1000 permutations. Line, treatment, and their interaction were evaluated for their effect on the observed metabolomic profiles.  The best fit model was chosen based on permutation statistics (permuted F scores), coefficient of determination, deviance metrics, and goodness of fit metrics.  

To examine relationships between labeled metabolites and lines, heirarchical cluster analysis was used to group compounds with similar abundances across lines.  Indicator species analysis (multi-level pattern analysis) was then used to explore associations of each labeled compound with lines and treatment using Pearson's Phi coefficient of association as the metric. 

To examine differences in abundance of individual labeled compounds, a metabolome wide association study approach was taken where Wilcoxon tests were applied to each compound by line to evaluate differences in compound abundance between uninoculated plants without nematodes and inoculated plants infected by nematodes.  Resultant P values were corrected for the false discovery rate using the Benjami and Hochberg method.  

To further explore the effect of individual compounds on nematode abundance, compounds of interest from the indicator species analysis and metabolome wide association were evaluated for their relationship to observed nematode population levels.  To do so, abundances and nematode counts were normalized by line to account for differences between genotypes then evaluated with linear models to determine the effect of normalized compound abundance on normalized nematode presence. Model fits were evaluated with information criteria, residual examination, model significance, and coefficient of determination.  


\subsubsection{Data Management}

Raw LC/mass spectrometry data were uploaded to metabolomics workbench.  All analysis on the raw data was conducted in R version 3.5.2 using RStudio as an IDE (with Vim keybindings).  A full list of packages used to facilitate the analysis can be found in Appendix E.  All code, including dockerfiles and manuscript documentation, is available on GitHub (...).  








Guidelines can be included for standard research article sections, such as this one. 

\lipsum[3]

\section{Some \LaTeX{} Examples}
\label{sec:examples}

Use section and subsection commands to organize your document. \LaTeX{} handles all the formatting and numbering automatically. Use ref and label commands for cross-references.

\subsection{Figures and Tables}

Use the table and tabular commands for basic tables --- see \TABLE{example}, for example. 

You can upload a figure (JPEG, PNG or PDF) using the project menu. To include it in your document, use the \verb|\includegraphics| command as in the code for \FIG{view}. 

For a half-width figure or table with text wrapping around it, use 

\begin{verbatim}
\begin{wrapfigure}{l}{.46\textwidth}
  \includegraphics[width=\hsize]{...}
  \caption{...}\label{...}
\end{wrapfigure}
\end{verbatim}
%
as in \FIG{halfwidth}. For tables:

\begin{verbatim}
\begin{wraptable}{l}{.46\textwidth}{
  \begin{tabular}{...}
  ...
  \end{tabular}}
  \caption{...}\label{...}
\end{wraptable}
\end{verbatim}

Be careful with these, though, as they may behave strangely near page boundaries, sectional headings, or in the neighbourhood of lists or too many floats.

Labels for main videos can be added with \verb|\video| e.g.

\video{Ths is a description of a main video.}\label{video:mv1}
\video{Another!}

Labels for video supplements can be added within \texttt{figure} environments, after the \texttt{caption}, using the \verb|\videosupp| command: see \VIDEOSUPP[view]{sv1} for an example.

If you use the following prefixes for your \verb|\label|:
%
\begin{description}
\item[Figures] \texttt{fig:}, e.g.~\verb|\label{fig:view}|
\item[Figure Supplements] \texttt{figsupp:}, e.g.~\verb|\label{figsupp:sf1}|\\
(we'll assume \texttt{figsupp:sf1} is a figure supplement of \texttt{fig:view} in our example)
\item[Figure source data] \texttt{figdata:}, e.g.~\verb|\label{figdata:first}|
\item[Videos] \texttt{video:}, e.g.~\verb|\label{video:mv1}|
\item[Video supplements] \texttt{videosupp:}, e.g.~\verb|\label{videosupp:sv1}|
\item[Tables] \texttt{tab:}, e.g.~\verb|\label{tab:example}|
\item[Equations] \texttt{eq:}, e.g.~\verb|\label{eq:CLT}|
\item[Boxes] \texttt{box:}, e.g.~\verb|\label{box:simple}|
\end{description}
%
you can then use the convenience commands \verb|\FIG{view}|, \verb|\FIGSUPP[view]{sf1}|, \verb|\TABLE{example}|, \verb|\EQ{CLT}|, \verb|\BOX{simple}|, \verb|\FIGDATA[view]{first}|, \verb|\VIDEO{mv1}| and \verb|{\VIDEOSUPP}[view]{sv1}| \emph{without} the label prefixes, to generate cross-references \FIG{view}, \FIGSUPP[view]{sf1},  \TABLE{example}, \EQ{CLT}, \BOX{simple}, \FIGDATA[view]{first}, \VIDEO{mv1} and \VIDEOSUPP[view]{sv1}. Alternatively, use \verb|\autoref| with the full label, e.g.~\autoref{first:app} (although this may not work correctly for figures and tables in the appendices or boxes nor supplements at present).

Really wide figures or tables, that take up the entire page, including the gutter space: use \verb|\begin{fullwidth}...\end{fullwidth}| as in \FIG{fullwidth}. And sometimes you may want to use feature boxes like \BOX{simple}.

\begin{wrapfigure}{l}{.46\textwidth}
\includegraphics[width=\hsize]{frog}
\caption{A half-columnwidth image using wrapfigure, to be used sparingly. Note that using a wrapfigure before a sectional heading, near other floats or page boundaries is not recommended, as it may cause interesting layout issues. Use the optional argument to wrapfigure to control how many lines of text should be set half-width alongside it.}
\label{fig:halfwidth}
\end{wrapfigure}

Some filler text to sit alongside the half-width figure. \lipsum[1-2]

\begin{figure}
\begin{fullwidth}
\includegraphics[width=0.95\linewidth]{elife-18156-fig2}
\caption{A very wide figure that takes up the entire page, including the gutter space.}
\label{fig:fullwidth}
\figsupp{There is no limit on the number of Figure Supplements for any one primary figure. Each figure supplement should be clearly labelled, Figure 1--Figure Supplement 1, Figure 1--Figure Supplement 2, Figure 2--Figure Supplement 1 and so on, and have a short title (and optional legend). Figure Supplements should be referred to in the legend of the associated primary figure, and should also be listed at the end of the article text file.}{\includegraphics[width=5cm]{frog}}
\end{fullwidth}
\end{figure}

\subsection{Citations}

LaTeX formats citations and references automatically using the bibliography records in your .bib file, which you can edit via the project menu. Use the \verb|\cite| command for an inline citation, like \cite{Aivazian917}, and the \verb|\citep| command for a citation in parentheses \citep{Aivazian917}. The LaTeX template uses a slightly-modified Vancouver bibliography style. If your manuscript is accepted, the eLife production team will re-format the references into the final published form. \emph{It is not necessary to attempt to format the reference list yourself to mirror the final published form.} Please also remember to \textbf{delete the line} \verb|\nocite{*}| in the template just before \verb|\bibliography{...}|; otherwise \emph{all} entries from your .bib file will be listed! 

\begin{featurebox}
\caption{This is an example feature box}
\label{box:simple}
This is a feature box. It floats!
\medskip

\includegraphics[width=5cm]{example-image}
\featurefig{`Figure' and `table' captions in feature boxes should be entered with \texttt{\textbackslash featurefig} and \texttt{\textbackslash featuretable}. They're not really floats.}

\lipsum[1]
\end{featurebox}

\subsection{Mathematics}

\LaTeX{} is great at typesetting mathematics. Let $X_1, X_2, \ldots, X_n$ be a sequence of independent and identically distributed random variables with $\text{E}[X_i] = \mu$ and $\text{Var}[X_i] = \sigma^2 < \infty$, and let
\begin{equation}
\label{eq:CLT}
S_n = \frac{X_1 + X_2 + \cdots + X_n}{n}
      = \frac{1}{n}\sum_{i}^{n} X_i
\end{equation}
denote their mean. Then as $n$ approaches infinity, the random variables $\sqrt{n}(S_n - \mu)$ converge in distribution to a normal $\mathcal{N}(0, \sigma^2)$.

\lipsum[3] 

\begin{figure}
\includegraphics[width=\linewidth]{elife-13214-fig7}
\caption{A text-width example.}
\label{fig:view}
%% If the optional argument in the square brackets is "none", then the caption *will not appear in the main figure at all* and only the full caption will appear under the supplementary figure at the end of the manuscript.
\figsupp[Shorter caption for main text.]{This is a supplementary figure's full caption, which will be used at the end of the manuscript.}{\includegraphics[width=6cm]{frog}}\label{figsupp:sf1}
\figsupp{This is another supplementary figure.}{\includegraphics[width=6cm]{frog}}
\videosupp{This is a description of a video supplement.}\label{videosupp:sv1}
\figdata{This is a description of a data source.}\label{figdata:first}
\figdata{This is another description of a data source.}\label{figdata:second}
\end{figure}

\subsection{Other Chemistry Niceties}

You can use commands from the \texttt{mhchem} and \texttt{siunitx} packages. For example: \ce{C32H64NO7S}; \SI{5}{\micro\metre}; \SI{30}{\degreeCelsius}; \SI{5e-17}{\Molar}

\subsection{Lists}

You can make lists with automatic numbering \dots

\begin{enumerate}
\item Like this,
\item and like this.
\end{enumerate}
\dots or bullet points \dots
\begin{itemize} 
\item Like this,
\item and like this.
\end{itemize}
\dots or with words and descriptions \dots
\begin{description}
\item[Word] Definition
\item[Concept] Explanation
\item[Idea] Text
\end{description}

Some filler text, because empty templates look really poorly. \lipsum[1]


\section{Acknowledgments}

Additional information can be given in the template, such as to not include funder information in the acknowledgments section.

\nocite{*} % This command displays all refs in the bib file. PLEASE DELETE IT BEFORE YOU SUBMIT YOUR MANUSCRIPT!
\bibliography{elife-sample}

%%%%%%%%%%%%%%%%%%%%%%%%%%%%%%%%%%%%%%%%%%%%%%%%%%%%%%%%%%%%
%%% APPENDICES
%%%%%%%%%%%%%%%%%%%%%%%%%%%%%%%%%%%%%%%%%%%%%%%%%%%%%%%%%%%%

\appendix
\begin{appendixbox}
\label{first:app}
\section{Firstly}
\lipsum[1]

%% Sadly, we can't use floats in the appendix boxes. So they don't "float", but use \captionof{figure}{...} and \captionof{table}{...} to get them properly caption.
\begin{center}
\includegraphics[width=\linewidth,height=7cm]{frog}
\captionof{figure}{This is a figure in the appendix}
\end{center}

\section{Secondly}

\lipsum[5-8]

\begin{center}
\includegraphics[width=\linewidth,height=7cm]{frog}
\captionof{figure}{This is a figure in the appendix}
\end{center}

\end{appendixbox}

\begin{appendixbox}
\includegraphics[width=\linewidth,height=7cm]{frog}
\captionof{figure}{This is a figure in the appendix}
\end{appendixbox}
\end{document}
